%
% examples.tex
%
% Copyright (C) 2023 UFSC.
%
% DOCUMENTATION-TEMPLATE
%
% This work is licensed under the Creative Commons Attribution-ShareAlike 4.0
% International License. To view a copy of this license,
% visit http://creativecommons.org/licenses/by-sa/4.0/.
%         
    
\chapter{High-level examples}\label{chp:chapter_ex}

    Welcome to the chapter that will help you put into practice all the lessons you have learned so far. Here you will find some high-level examples. It is important to point out that the code that you are going to develop during the course won't have pre-built functions. So, you are going to develop low-level code. All code is written in C, so we hope you have already read the previous chapters to understand how to handle all the NEORV32's registers, as well as to programming in C.

    \autoref{sec:section_ex.1}:
    In this section we have an example of a GPIO implementation in the NEORV32.
        
    \autoref{sec:section_ex.2}:
    In this section we have an example of a 7-segment display implementation in the NEORV32.

    \autoref{sec:section_ex.2}:
    In this section we have an example of a 7-segment display implementation in the NEORV32.

    \autoref{sec:section_ex.3}:
    In this section we have an example of a LCD display implementation in the NEORV32.

    \autoref{sec:section_ex.4}:
    In this section we have an example of a interrupts implementation in the NEORV32.
    
    \autoref{sec:section_ex.5}:
    In this section we have an example of a timer/counter implementation in the NEORV32.

    \autoref{sec:section_ex.5}:
    In this section we have an example of a serial UART implementation in the NEORV32.

    \section{GPIO}\label{sec:section_ex.1}

        In this section \autoref{sec:section_ex.1} we will understand a simple example of using a GPIO.
        The idea of the implementation is to make a counter using the red LEDs starting from LEDR[7] to LEDR[0], and as seen in \autoref{sec:section_ex.1} we only have 8 pins available for red LEDs.

    
        \begin{lstlisting}[style=mystyle_c, language=c, breaklines]
            /************************************************************//**
             * @file demo_blink_led/main.c
             * @author J.C.E. Barcellos
             * @brief A simple GPIO application.
             ***************************************************************/
            #include <neorv32.h>
            /************************************************************//**
             * @brief Shows an incrementing 8-bit counter on LEDR[7..0].
             *
             * @note This program requires the GPIO controller to be synthesized.
             *
             * @return Will never return.
             ***************************************************************/
            int main() {
            
            	int cnt = 0; // counter
            	
            	while (1) {
            		neorv32_gpio_port_set((cnt++ & 0xFF) << 9); // increment counter and mask for lowest 8 bits
            		neorv32_cpu_delay_ms(500); // wait 500ms
            	}
            	
            	return 0; // this should never be reached
            }
    \end{lstlisting}

    So let's start with a declaration of the variable ``cnt'', with initial data of 0. 
    
    Further down in a while loop we use the function ``neorv32\_gpio\_port\_set()'' from the GPIO library to link the parameter to the output (the LEDs). In this case, the parameter of this function will be the variable ``cnt'' being incremented, plus a logical AND operation with 0xFF (which in binary would be ``11111111''), all being shifted 9 bits.
   
    Now let's understand what this all means: 
    \& 0xFF would be a mask created for cnt, where only the 8 LSB bits are used. When these 8-bit values resulting from the mask are dumped into port\_set, the green LEDs will turn on instead of the red ones, but why does this happen? This is due to the fact that the GPIO\_o pins start with LEDGs that go from bit 0 to 8 (available in \autoref{tab:ledg_o}), so it is necessary to shift 9 bits so that the red LEDs to be set.
    
    \section{7-segment Display} \label{sec:section_ex.2}

        In this section \autoref{sec:section_ex.2} we will learn how to implement a 7-segment display. Make it display values from '0' to 'F' periodically. 

            \begin{lstlisting}[style=mystyle_c, language=c, breaklines]
                /************************************************************//**
                 * @file demo_lcd/main.c
                 * @author J.C.E. Barcellos
                 * @brief A simple 7-segment display application.
                 ***************************************************************/
                #include <neorv32.h>
                #include <stdint.h>
                #include <stdio.h>
                #include <string.h>
                /************************************************************//**
                 * @brief Prints 0 to 15 in hex format into the 7-segment display.
                 *
                 * @note This program requires the GPIO controller to be synthesized.
                 *
                 * @return Will never return.
                 ***************************************************************/
                int main() {
                
                	char display_chars[] = "0123456789ABCDEF"; // characters to be displayed in sequence
                	int num_chars = sizeof(display_chars) - 1; // -1 to exclude the null terminator
                
                	while (1) {
                		for (int i = 0; i < num_chars; i++) {
                			neorv32_display_write(CHANNEL_0, &display_chars[i]); // display each charater to the 7-segment display
                			neorv32_cpu_delay_ms(500); // wait 1s
                		}
                		neorv32_display_clear(CHANNEL_0); // clears the 7-segment display
                		neorv32_cpu_delay_ms(2000); // wait 2s
                	}
                
                	return 0; // this should never be reached
                }       
        \end{lstlisting}

        Now we are going to implement a 7-segment display. We create a variable of type character, ``display\_chars'' which will be our sequence of characters that will be shown on the display. Another variable will be of type integer which will be the size of the variable ``display\_chars'' - 1, being the maximum bit the sequence will have.
    
        The function ``neorv32\_display\_write()'' will be responsible for sending the character to a converter and then to the display. The first parameter will be in charge of shifting the character to the pin available for the 7-segment display, in the case of CHANNEL\_0, it will be a 16-bit shift. And so, each increment of ``i'' and each character of display\_chars[] will be sent to the display.
        
        Finally, the display will be cleared as soon as ``i'' is bigger than ``num\_chars'' and a delay of 2000 ms will be executed.
        
        Remember that you can see any details about the functions of the library responsible for the LCD Display at \autoref{.c:-7-segments} and \autoref{.h:7-segments}.

            
    \section{LCD Display} \label{sec:section_ex.3}

        In this section \autoref{sec:section_ex.3} we will learn how to implement an LCD Display. The goal is simple, write a message and have it appear on the display.

            \begin{lstlisting}[style=mystyle_c, language=c, breaklines]
                /************************************************************//**
                 * @file demo_lcd/main.c
                 * @author J.C.E. Barcellos
                 * @brief A simple LCD application.
                 ***************************************************************/
                #include <neorv32.h>
                #include <stdint.h>
                #include <stdio.h>
                #include <string.h>
                /************************************************************//**
                 * @brief Prints a series of sentences into the LCD.
                 *
                 * @note This program requires the GPIO controller to be synthesized.
                 *
                 * @return Will never return.
                 ***************************************************************/
                int main() {
                
                	// turn on the LCD with a blinking cursor
                	neorv32_lcd_display_on(); 
                
                	// writes the first sentence to the LCD
                	neorv32_lcd_write("Hello, guys!"); 
                	neorv32_cpu_delay_ms(2000);
                
                	// clear the LCD and return the cursor to its origin
                	neorv32_lcd_clear_display(); 
                	neorv32_lcd_return_to_origin(); 
                	neorv32_cpu_delay_ms(2000);
                
                	// writes the second and last sentence to the LCD
                	neorv32_lcd_write("How are you?"); 
                
                	// go to sleep mode
                    while(1) {
                        neorv32_cpu_sleep();
                    }
                	
                	return 0; // this should never be reached
                }        
        \end{lstlisting}

            We will start the code turn on the LCD display with the function ``neorv32\_lcd\_display\_on()''. Next we will use ``neorv32\_lcd\_write()'' to convert the chosen message into a character and write it to the LCD display. So when we are done sending the message we need to clear the LCD and return the cursor to origin with the respective functions: ``neorv32\_lcd\_clear\_display()'' and ``neorv32\_lcd\_return\_to\_origin()''. Then it will write a message again and go into sleep mode.
            Remember that you can see any details about the functions of the library responsible for the LCD Display at \autoref{.c:lcd} and \autoref{.h:lcd}.
    
    \section{Interrupts} \label{sec:section_ex.4}
    
        In this section \autoref{sec:section_ex.4} we learned about interrupts and now make an external interrupt controller using the libraries available in neorv32.

            \begin{lstlisting}[style=mystyle_c, language=c, breaklines]
                /************************************************************//**
                 * @file demo_xirq/main.c
                 * @author J.C.E. Barcellos
                 * @brief A simple external interruption application.
                 ***************************************************************/
                #include <neorv32.h>
                #include <string.h>
                
                // defines
                #define LEDR_0 0x09
                
                /************************************************************//**
                 * @brief XIRQ handler.
                 ***************************************************************/
                void button_xirq_handler(void) {
                    neorv32_gpio_pin_toggle(LEDR_0); // toggle the LEDR[0]
                }
                
                /************************************************************//**
                 * @brief Toggles a LED every every time a button is pressed.
                 *
                 * @note This program requires the XIRQ and the GPIO controller to be synthesized.
                 *
                 * @return Will never return.
                 ***************************************************************/
                int main() {
                
                    // initialize the runtime environment, to handling all CPU's traps
                    neorv32_rte_setup();
                
                    // check if XIRQ controller was synthesized.
                    if (neorv32_xirq_available() == 0) {
                        neorv32_gpio_pin_set(0x01);
                    }
                
                    // initialize the XIRQ controller
                    if (neorv32_xirq_setup() != 0) {
                        neorv32_gpio_pin_set(0x02);
                    }
                
                    // install handler functions for XIRQ 
                    if (neorv32_xirq_install(0, button_xirq_handler) != 0) {
                        neorv32_gpio_pin_set(0x03);
                    }
                
                    // enable XIRQ interrupts
                    neorv32_xirq_global_enable();
                
                    // enable machine-mode interrupts
                    neorv32_cpu_csr_set(CSR_MSTATUS, 1 << CSR_MSTATUS_MIE);
                
                    while(1){
                        neorv32_cpu_sleep();
                    }
                
                    return 0;
                }        
        \end{lstlisting}

        We will start by adding a ``define'' with the LEDR that is responsible for lighting up when there is an interruption and this will be included in a function that links this LEDR with a GPIO.
        So next we use the function ``neorv32\_rte\_setup'' to initialize the runtime environment. We check if the interrupt controller is synthesized. If so we set ``neorv32\_gpio\_pin\_set(0x01)'' in the GPIO.  Also if the controller is not initialized we set ``neorv32\_gpio\_pin\_set(0x02)'' to initialize it. We install the handler functions for XIRQ by setting ``neorv32\_gpio\_pin\_set(0x03)''. To make the handlers interrupt enable ``neorv32\_xirq\_global\_enable()''. Then we make machine-mode interrupts enable. 
        Remember that every library responsible for XIRQ can be found in the neorv32 files.
    
    \section{Timer/Counter} \label{sec:section_ex.5}

            In this section \autoref{sec:section_ex.5} we learned about timers/counters and now in a similar way to the external interrupt controller, so make a timer/counter application, with 1 second delay and using the libraries available in neorv32.

            \begin{lstlisting}[style=mystyle_c, language=c, breaklines]
                /************************************************************//**
                 * @file demo_mtime/main.c
                 * @author J.C.E. Barcellos
                 * @brief A simple timer application.
                 ***************************************************************/
                #include <neorv32.h>
                
                // defines
                #define DELAY_S 1
                
                /************************************************************//**
                 * @brief MTIME IRQ handler.
                 ***************************************************************/
                void mtime_irq_handler(void) {
                    static int cnt = 0; // counter
                
                    // update the time for the next interrupt
                    neorv32_mtime_set_timecmp(neorv32_mtime_get_timecmp() + DELAY_S*(NEORV32_SYSINFO->CLK / 2));
                
                    // increment counter and mask for lowest 8 bits
                	neorv32_gpio_port_set((cnt++ & 0xFF)<<9); 
                }
                
                /************************************************************//**
                 * @brief increments a 8-bit counter on LEDR[7..0] using the machine timer interrupt.
                 *
                 * @note This program requires the MTIME and the GPIO controller to be synthesized.
                 *
                 * @return Will never return.
                 ***************************************************************/
                int main() {
                
                    // initialize the runtime environment, to handling all CPU's traps
                    neorv32_rte_setup();
                
                    // check if MTIME controller was synthesized.
                    if (neorv32_mtime_available() == 0) {
                        neorv32_gpio_pin_set(0x01);
                    }
                
                    // install handler functions for MTIME 
                    if (neorv32_rte_handler_install(RTE_TRAP_MTI, mtime_irq_handler) != 0) {
                        neorv32_gpio_pin_set(0x02);
                    }
                
                    // configure the time for the first interrupt
                    neorv32_mtime_set_timecmp(neorv32_mtime_get_time() + DELAY_S*(NEORV32_SYSINFO->CLK / 2));
                
                    // enable interrupt
                    neorv32_cpu_csr_set(CSR_MIE, 1 << CSR_MIE_MTIE); // enable MTIME interrupt
                    neorv32_cpu_csr_set(CSR_MSTATUS, 1 << CSR_MSTATUS_MIE); // enable machine-mode interrupts
                
                    // go to sleep mode
                    while(1) {
                        neorv32_cpu_sleep();
                    }
                
                    return 0; // this should never be reached
                }            
        \end{lstlisting}

        For the timer/counter we set a delay. We create a function responsible for controlling the timer ``MTIME IRQ'' that will update the time for the next interrupt and increment the counter to the least significant 8 bits.

        In the main we start the runtime environment, to handle all CPU's traps, we check if the controller is synthesize. We install the handler functions for MTIME and set the time for the first interrupt, making the time now plus 1 second ((NEORV32\_SYSINFO->CLK / 2) is equal to 1). We make the interrupt enable and the machine-mode interrupts. 


    
    \section{Serial UART} \label{sec:section_ex.6}

        In this section \autoref{sec:section_ex.6} we learned about UART and now do the application using UART, that when we send a certain message, we will receive a specific message back from the microprocessor.

            \begin{lstlisting}[style=mystyle_c, language=c, breaklines]
                /************************************************************//**
                 * @file demo_uart/main.c
                 * @author J.C.E. Barcellos
                 * @brief A simple UART application.
                 ***************************************************************/
                #include <neorv32.h>
                #include <stdio.h>
                #include <string.h>
                
                // defines
                #define BAUD_RATE 19200
                
                // prototypes
                void uart_irq_handler(void);
                
                /************************************************************//**
                 * @brief Every time it receives the question "Are you alright?" it will respond with "I'm alive".
                 *
                 * @note This program requires the UART and the GPIO controller to be synthesized.
                 *
                 * @return Will never return.
                 ***************************************************************/
                int main() {
                
                    char buffer[50]; // buffer for the received message
                
                    // initialize the UART with default baud rate
                    neorv32_uart0_setup(BAUD_RATE, 0);
                
                    // initialize the runtime environment, to handling all CPU's traps
                    neorv32_rte_setup();
                
                    // check if UART controller was synthesized.
                    if (neorv32_uart0_available() == 0) {
                        neorv32_gpio_pin_set(0x01);
                    }
                
                
                    while(1) {
                        // send a message for the user and waits for a response
                        neorv32_uart0_printf("Hi! \n");
                        neorv32_uart0_scan(buffer, 50, 1);
                        neorv32_uart0_printf("\n");
                
                        // compares the response with the expected one
                        if (!strcmp(buffer, "Are you alright?")) {
                            neorv32_uart0_printf("\n I'm alive! \n");
                            neorv32_uart0_printf("=) \n");
                            neorv32_uart0_printf("NEORV32 left the conversation... \n");
                        }
                
                        // go to sleep mode
                        // but you could remove it, to start a new interaction
                        while(1) {
                            neorv32_cpu_sleep();
                        }
                    }
                
                    return 0; // this should never be reached
                }
        \end{lstlisting}

        First, a buffer is created for incoming messages. We will start the UART with the usual settings and also the runtime environment, to handle all CPU's traps. We check if the UART is synthesized. So we print out a message and send it to the UART via ``neorv32\_uart0\_scan(buffer, 50, 1)''. If the message is ``Are you alright?'' the microprocessor will reply ``I'm alive!'' and ``NEORV32 left the conversation... ''.
        And finally it goes into sleep mode unless you want to start a new interaction.
        
    

