\chapter{Introduction}

RISC-V is an open-source instruction set architecture (ISA) designed for computer processors. It stands for ``Reduced Instruction Set Computing - Version 5.0''. RISC-V is different from proprietary ISAs like x86 (Intel) or ARM because it is an open standard that can be freely used, modified, and implemented by anyone without requiring license fees or restrictions.

The RISC-V ISA is based on a simple, modular, and extensible design philosophy. It aims to provide a flexible foundation for a wide range of computing devices, from small embedded systems to large-scale servers. The instruction set is designed to be efficient, easy to understand, and conducive to efficient hardware implementation.

RISC-V supports several standard extensions, which provide additional functionality beyond the base instruction set. These extensions include integer multiplication and division, floating-point arithmetic, atomic memory operations, vector processing (SIMD), and more. The modular nature of RISC-V allows implementers to choose the extensions that best suit their specific requirements.

One of the key advantages of RISC-V is its open nature. It enables academic researchers, industry professionals, and hobbyists to contribute to the development and improvement of the architecture. This openness has led to a growing ecosystem around RISC-V, including a wide range of compatible processors, development boards, software tools, and libraries.

RISC-V has gained significant attention and adoption in recent years, particularly in areas such as Internet of Things (IoT), embedded systems, and academic research. It has also seen interest from industry giants like NVIDIA, Western Digital, and SiFive, who have developed RISC-V-based products or contributed to the development of the ecosystem.

In this document, different aspects of RiSC-V are studied, dividing the document into chapters. In \autoref{ch:chapter_riscv_prof}, the RISC-V architecture is analyzed and explained, explaining each block and the operation of the pipeline, as well as a brief analysis of the set of instructions. In \autoref{ch:chapter_riscv_stud}, the RISC-V is explained in a simple and concise way to make it easier for beginning students to understand. Then, in \autoref{chp:chapter_cl}, an introduction to the C language is made using a micro-controller with RISC-V architecture, where some examples are presented. In \autoref{chp:chapter_s} a representation of the memory allocation mode used by NEORV32 is made. Then, in \autoref{chp:chapter_p}, the peripherals of the microprocessor based on RISC-V NEORV32 are presented. In \autoref{chp:chapter_rec}, the procedure for configuring the FPGA to operate with NEORV32 and how to install the toolchain is explained. Then, in \autoref{ch:chapter_comp} the first compilation and execution of a program is made using the presented setup. Finally, in \autoref{chp:chapter_ex}, more advanced examples are presented in C language using the RISC-V architecture.



