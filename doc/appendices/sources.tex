%
% sources.tex
%
% Copyright (C) 2023 UFSC.
%
% DOCUMENTATION-TEMPLATE
%
% This work is licensed under the Creative Commons Attribution-ShareAlike 4.0
% International License. To view a copy of this license,
% visit http://creativecommons.org/licenses/by-sa/4.0/.
%


\chapter{Sources}

        \section{LCD}\label{.c:lcd}
            \begin{lstlisting}[style=mystyle_c, language=c, breaklines]
                /************************************************************//**
                 * @author J.C.E. Barcellos
                 * @author Y.A. Guevara
                 * @author R.Q. Do O
                 * @file neorv32_lcd.c
                 * @brief Liquid Crystal Display (LCD) HW driver source file.
                 *
                 * @note These functions should only be used if the GPIO unit was synthesized (IO_GPIO_EN = true).
                 ***************************************************************/
                #include <stdint.h>
                #include <stdio.h>
                #include <string.h>
                
                #include "neorv32.h"
                #include "neorv32_gpio.h"
                #include "neorv32_lcd.h"
                
                
                /************************************************************//**
                 * @brief Turn on the display, with a blinking cursor.
                 ***************************************************************/
                void neorv32_lcd_display_on(void){
                    
                    neorv32_gpio_port_set(0);
                
                    neorv32_gpio_port_set((uint64_t)0x08 << 60 );    // Result in 0x 8000 0000 0000 0000 // 0b 1000 0000 0000 0000 0000 0000 0000 0000 0000 0000 0000 0000 0000 0000 0000 0000
                    neorv32_cpu_delay_ms(10);
                    neorv32_gpio_port_set((uint64_t)0x0838 << 60 );    // Result in 0x 8380 0000 0000 0000 // 0b 1000 0011 1000 0000 0000 0000 0000 0000 0000 0000 0000 0000 0000 0000 0000 0000
                    neorv32_cpu_delay_ms(10);
                    neorv32_gpio_port_set((uint64_t)0x0938 << 60 );    // Result in 0x 9380 0000 0000 0000 // 0b 1000 0011 1000 0000 0000 0000 0000 0000 0000 0000 0000 0000 0000 0000 0000 0000
                    neorv32_cpu_delay_ms(10);
                    neorv32_gpio_port_set((uint64_t)0x0838 << 60 );
                    neorv32_cpu_delay_ms(10);
                    neorv32_gpio_port_set((uint64_t)0x080F << 60 );
                    neorv32_cpu_delay_ms(10);
                    neorv32_gpio_port_set((uint64_t)0x090F << 60 );
                    neorv32_cpu_delay_ms(10);
                    neorv32_gpio_port_set((uint64_t)0x080F << 60 );
                    neorv32_cpu_delay_ms(10);
                }
                
                /************************************************************//**
                 * @brief Clear the display
                 ***************************************************************/
                void neorv32_lcd_clear_display(void){
                
                    neorv32_gpio_port_set((uint64_t)0x0801 << 60  );
                    neorv32_cpu_delay_ms(10);
                    neorv32_gpio_port_set((uint64_t)0x0901 << 60  );
                    neorv32_cpu_delay_ms(10);
                }
                
                /************************************************************//**
                 * @brief Return the display's cursor to the origin.
                 ***************************************************************/
                void neorv32_lcd_return_to_origin(void){
                
                    neorv32_gpio_port_set((uint64_t)0x0802 << 60 );
                    neorv32_cpu_delay_ms(10);
                    neorv32_gpio_port_set((uint64_t)0x0902 << 60 );
                    neorv32_cpu_delay_ms(10);
                }
                
                /************************************************************//**
                 * @brief Move cursor to the left.
                 *
                 * @param[in] n is the number of times the cursor is going to the left.
                 ***************************************************************/
                void neorv32_lcd_move_cursor_left(int n){
                
                    int i = 0; 
                
                    for(i = 0; i<=n; i++){
                        neorv32_gpio_port_set((uint64_t)0x081 << 60 );
                        neorv32_cpu_delay_ms(10);
                        neorv32_gpio_port_set((uint64_t)0x091 << 60 );
                        neorv32_cpu_delay_ms(10);
                    }
                }
                
                /************************************************************//**
                 * @brief Move cursor to the right.
                 *
                 * @param[in] n is the number of times the cursor is going to the right.
                 ***************************************************************/
                void neorv32_lcd_move_cursor_right(int n){
                
                    int i = 0; 
                
                    for(i = 0; i<=n; i++){
                        neorv32_gpio_port_set((uint64_t)0x0814 << 60 );
                        neorv32_cpu_delay_ms(10);
                        neorv32_gpio_port_set((uint64_t)0x0914 << 60 );
                        neorv32_cpu_delay_ms(10);
                    }
                }
                
                /************************************************************//**
                 * @brief Write a sentence in the display.
                 *
                 * @param[in] message is the sentence to be written to the display.
                 ***************************************************************/
                void neorv32_lcd_write(const char *message){
                
                    size_t word_len = strlen(message);
                    uint64_t h = 0;
                    uint64_t l = 0;
                
                    for (size_t i = 0; i <= word_len; i++) {
                        h = (message[i] & 0xF0) >> 4;
                        l = (message[i] & 0x0F);
                
                        neorv32_gpio_port_set((uint64_t)0x06  <<  60 | h  <<  56 | l  <<  52);
                        neorv32_cpu_delay_ms(10);
                        neorv32_gpio_port_set((uint64_t)0x0D  <<  60 | h  <<  56 | l  <<  52);
                        neorv32_cpu_delay_ms(10);
                    }
                
                }
        \end{lstlisting}

        \section{Display 7-segments}\label{.c:-7-segments}
        \begin{lstlisting}[style=mystyle_c, language=c, breaklines]
            /************************************************************//**
             * @author J.C.E. Barcellos
             * @author Y.A. Guevara
             * @author R.Q. Do O
             * @file neorv32_display.c
             * @brief 7-Segment Display HW driver source file.
             *
             * @note These functions should only be used if the GPIO unit was synthesized (IO_GPIO_EN = true).
             ***************************************************************/
            #include <stdint.h>
            #include <stdio.h>
            #include <string.h>
            
            #include "neorv32.h"
            #include "neorv32_gpio.h"
            #include "neorv32_display.h"
            
            /************************************************************//**
             * @brief Clear the display
             ***************************************************************/
            void neorv32_display_clear(int channel){
            
                neorv32_gpio_port_set((uint64_t)0x40 << channel);
            }
            
            /************************************************************//**
             * @brief Converting the character to a binary value, to display into the 7-segment display
             *
             * @param[in] value is the character to be written to the display.
             ***************************************************************/
            uint8_t neorv32_display_convert(char *value){
            
                uint8_t p = 0;
            
                switch(*value){
                    case '0':
                        p = 0b0111111; // "0"          
                        break;
                    case '1':
                        p = 0b0000110; // "1"
                        break;         
                    case '2':
                        p = 0b1011011; // "2"
                        break;         
                    case '3':
                        p = 0b1001111; // "3"         
                        break;
                    case '4':
                        p = 0b1100110; // "4"         
                        break;
                    case '5':
                        p = 0b1101101; // "5"         
                        break;
                    case '6':
                        p = 0b1111101; // "6"        
                        break;
                    case '7':
                        p = 0b0000111; // "7"
                        break;
                    case '8':
                        p = 0b1111111; // "8"
                        break;
                    case '9':
                        p = 0b1101111; // "9"
                        break;
                    case 'A':
                        p = 0b1110111; // 'A'
                        break;
                    case 'B':
                        p = 0b1111100; // 'b'
                        break;
                    case 'C':
                        p = 0b0111001; // 'C'
                        break;
                    case 'D':
                        p = 0b1011110; // 'd'
                        break;
                    case 'E':
                        p = 0b1111001; // 'E'
                        break;
                    case 'F':
                        p = 0b1110001; // 'F'
                        break;
                    case 'G':
                        p = 0b0111101; // 'G'
                        break;
                    case 'H':
                        p = 0b1110110; // 'H'
                        break;
                    case 'I':
                        p = 0b0000110; // 'I'
                        break;
                    case 'J':
                        p = 0b0001110; // 'J'
                        break;
                    case 'K':
                        p = 0b1110110; // 'K'  (same as 'H')
                        break;
                    case 'L':
                        p = 0b0111000; // 'L'
                        break;
                    case 'M':
                        p = 0b0000000; // 'M'  (no display)
                        break;
                    case 'N':
                        p = 0b1010100; // 'n'
                        break;
                    case 'O':
                        p = 0b0111111; // 'O'
                        break;
                    case 'P':
                        p = 0b1110011; // 'P'
                        break;
                    case 'Q':
                        p = 0b1100111; // 'q'
                        break;
                    case 'R':
                        p = 0b1010000; // 'r'
                        break;
                    case 'S':
                        p = 0b1101101; // 'S'
                        break;
                    case 'T':
                        p = 0b1111000; // 't'
                        break;
                    case 'U':
                        p = 0b0111110; // 'U'
                        break;
                    case 'V':
                        p = 0b0111110; // 'V'  (same as 'U')
                        break;
                    case 'W':
                        p = 0b0000000; // 'W'  (no display)
                        break;
                    case 'X':
                        p = 0b1110110; // 'X'  (same as 'H')
                        break;
                    case 'Y':
                        p = 0b1101110; // 'y'
                        break;
                    case 'Z':
                        p = 0b1011011; // 'Z'  (same as '2')
                        break;
                    case ' ':
                        p = 0b0000000; // ' '  (blank)
                        break;
                    case '-':
                        p = 0b1000000; // 37  '-'  
                        break;
                };
            
                return ~p;
            }
            
            
            /************************************************************//**
             * @brief Write a character in the display.
             *
             * @param[in] value is the character to be written to the display.
             ***************************************************************/
            void neorv32_display_write(int channel, char *value){
                
                neorv32_gpio_port_set((uint64_t)neorv32_display_convert(value) <<  channel);
            }
        \end{lstlisting}
